\section{Introduction}
\label{sec:intro}
%%

Query-driven techniques have become increasingly popular for exploring and visualizing scientific datasets. Statistical summaries such as mean, standard deviation, entropy and higher order moments computed from distributions of spatial sub-ranges are useful for feature extraction, uncertainty quantification, multi-resolution analysis, and data reduction, just to name a few. Useful local statistics are often derived from distributions commonly represented as histograms. In general, this type of queries is known as \emph{range distribution query}~\cite{martin13}, which returns the distribution of an axis-aligned query region of any arbitrary size. Traditionally, such queries are supported by running a sequential scan through the raw data. However, as the data size grows, frequently accessing large subsets of the raw data increases workload, which leads to slower query response time. This is because both the time and space complexity of such queries are proportional to the number of data points in the query region.

The importance of range distribution query is apparent from its wide use, directly or indirectly, in a number of visualization algorithms~\cite{Hixel11, uncertain3d11, qdv11} (explained in more detail in Section~\ref{sec:relwork}). Since the approach of scanning raw data does not scale well with data size, it is important to devise strategies to maintain the interactivity of a distribution query engine. In this paper, we propose to transform the raw data into a data structure which stores and uses pre-computed distributions to answer distribution query for any range in constant time. However, we need to address two major challenges to make this approach effective:
%%
\begin{packed_itemize}
%%
\item {\bf Query workload:} The workload of the query engine should be low and constant regardless of the query size and data size. The workload refers to a combination of the cost of accessing data in memory, the I/O cost associated with reading files from the disk, if needed, and also the communication cost of exchanging partial query results, if needed.
%%
\item {\bf Distribution storage:} The number of pre-computed distributions required to answer range distributions queries is as large as the number of data points. We must minimize the storage cost of the pre-computed distributions so that the query can be run on commodity desktops with moderate amount of storage. 
%%
\end{packed_itemize}

To bound the query workload, we adapt a data structure called integral distribution~\cite{integhist05}. It is an extension of the summed area table (SAT) proposed by Crow~\cite{SAT84} which, at each location of the data, stores the sum of the attribute values from the origin up to that location. SAT can compute the sum of an attribute in any rectangular region in constant time, regardless of the region size. Similarly, integral distribution can compute the distribution of any query region by accessing pre-computed distributions only at the corner points of the query region. Hence, the workload is bounded by the number of corner points which is same for any query of any size. 

The benefit, however, comes at the cost of huge storage cost. The total space required to store an integral histogram is in the order of $O(N*K)$ for $N$ data points and $K$ number of bins per histograms. This is prohibitively large even for data sets of moderate size. We address this storage issue by proposing a set of transformations to stored integral distributions to make them compression-friendly. After the proposed transformations - a decomposition followed by an indexing - are run on the integral distributions, any off-the-shelf compression technique can be employed to achieve much higher space saving. We propose the following transformations for integral distributions: First, we decompose them into a hierarchy of block distributions pertaining to power-of-two length sub-ranges. We propose a novel similarity-based indexing technique to be employed on these block distributions. These transformations lead to significant space saving and much faster query performance.

The rest of the paper is organized as follows. Section~\ref{sec:relwork} summarized the related work on management and storage of distribution data. The following Section~\ref{sec:distqueryframework} describes our framework including the transformations proposed to integral distribution, similarity-based indexing of distributions and the query mechanism. Section~\ref{sec:application} demonstrates how our technique benefits visualization applications. A detailed quantitative study of the framework is presented in Section~\ref{sec:analysis},  followed by discussion (Section~\ref{sec:discussion}) and conclusion (Section~\ref{sec:conclusion}). 
